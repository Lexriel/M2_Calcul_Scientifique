To realize something which seems simple at the first look becomes tricky when you want to parallelize it. You face problems and you must elaborate solutions using your knowledge, your own researchs and using the researchs of scientifics. This internship becomes me aware of the utility to be informed of the works and papers submitted by scientifics around the world. \\

To realize the Taylor shift by $1$ as fast as possible is a part of a huge and difficult work. When this will be done thanks to the work on the mix radix representation, we will be able to continue towards the isolation of the real roots of a univariate polynomial, but we don't forget that we will have others step like to consider rational then real polynomials, and so considering others works, other algorithms like the VCA algorithm which is the algorithm we want to use. \\

When you call several times the same procedure, even if the execution time of this procedure is very small, then you can have a very high execution time. For example, to run a Taylor shift with my code for a polynomial of size $2^{17}$ modulo a prime number cost 0.1s approximatively. To obtain the Taylor shift in $\mathbb{Z}$ need to call this for more than $300$ prime numbers, then finally we will have a cost of more than 30s for getting a Taylor shift in $\mathbb{Z}$. In the VCA algorithm, you use a lot this taylor shift, then you multiply your 30s again by a factor. Finally, even if you improve a little the procedures which are called several times, as the \textit{mul\_mod} procedure in my taylor shift code, you will have a huge gain of performances in you main code, that's why we do the maximum to parallelize everything can be parallelized in the best possible way.
