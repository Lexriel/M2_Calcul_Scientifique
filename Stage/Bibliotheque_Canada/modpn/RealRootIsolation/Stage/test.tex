\documentclass[10pt]{report}


\usepackage{latexsym}
\usepackage{amsmath}
\usepackage{amssymb}
\usepackage{amsfonts}
\usepackage{amsthm}
\usepackage{amscd}
\usepackage{epsfig}
\usepackage{verbatim}
\usepackage{fancybox}
\usepackage{moreverb}
\usepackage{graphicx}
\usepackage{psfrag}
\usepackage{hyperref}
\usepackage[all]{xy}
\usepackage[toc,page]{appendix}
\usepackage[subnum]{cases}
\usepackage{bm}
\usepackage{framed}
\usepackage{color}
\usepackage{dsfont}
\usepackage{textcomp}
\usepackage{graphicx}
\usepackage{url}
\usepackage[francais]{babel}
\usepackage[utf8]{inputenc}  
\usepackage[T1]{fontenc}
\usepackage[top=2cm, bottom=2cm, left=2cm, right=2cm]{geometry}


%\textheight 22cm    \textwidth 16cm
%\voffset = 0 cm
%\hoffset = 0 cm
%\marginparwidth = 0pt
%\oddsidemargin = 31pt
%one inch + \hoffset
%one inch + \voffset
%\oddsidemargin = 31pt
%\topmargin = 20pt
%\headheight = 12pt
%\headsep = 25pt
%\textheight = 592pt
%\textwidth = 390pt
%\marginparsep = 10pt
%\marginparwidth = 35pt
%\footskip = 30pt
%\marginparpush = 7pt
%\hoffset = 0pt
%\voffset = 0pt
%\paperwidth = 597pt
%\paperheight = 845pt


\newcommand{\C}{{\mathbb C}}
\newcommand{\R}{{\mathbb R}}
\newcommand{\N}{{\mathbb N}}
\newcommand{\Z}{{\mathbb Z}}
\newcommand{\Q}{{\mathbb Q}}
\newcommand{\T}{{\mathbb T}}
\newcommand{\E}{{\mathbb E}}
\newcommand{\di}{{\mathbb D}}
\newcommand{\Y}{{\mathbf Y}}
\newcommand{\D}{{\partial}}
\newcommand{\Cl}{{\mathcal C}}
\newcommand{\Pa}{{\mathcal P}}
\newcommand{\Flux}{{\mathcal F}}
\newcommand{\B}{{\mathfrak B}}
\newcommand{\M}{{\mathcal M}}
\newcommand{\dis}{{\mathcal D}}
\newcommand{\A}{{\mathcal{A}}}
\newcommand{\fin}{\rule{1ex}{1ex}}
\newcommand{\HRule}{\rule{\linewidth}{0.5mm}}
\renewcommand\labelitemi{\textbullet}

\newtheorem{theorem}{Theorem}
\newtheorem{lemma}[theorem]{Lemma}
\newtheorem{proposition}[theorem]{Proposition}
\newtheorem{corollary}[theorem]{Corollary}
\newtheorem{definition}[theorem]{Definition}
\newtheorem{example}[theorem]{Example}
\newtheorem{remark}[theorem]{Remark}
\newtheorem{notation}[theorem]{Notation}
\newtheorem{hypothesis}[theorem]{Hypothesis}


\renewcommand{\thesection}{\arabic{section}}
\renewcommand{\thelemma}{\thesection\arabic{lemma}}
\renewcommand{\theproposition}{\thesection\arabic{proposition}}
\renewcommand{\thetheorem}{\thesection\arabic{theorem}}
\renewcommand{\thecorollary}{\thesection\arabic{corollary}}
\renewcommand{\thedefinition}{\thesection\arabic{definition}}
\renewcommand{\theexample}{\thesection\arabic{example}}
\renewcommand{\theremark}{\thesection\arabic{remark}}
\renewcommand{\thenotation}{\thesection\arabic{notation}}
\renewcommand{\theequation}{\thesection.\arabic{equation}}
\def\commutatif{\ar@{}[rd]|{\circlearrowleft}}


\begin{document}

Let $b = (b_1 \ldots, b_s)$ be the $m = (m_1, \ldots, m_s)$-mixed radix 
representation of $n$ computed from the residue vector 
$x = (x_1, \ldots, x_s)$ via some program.
Thus we have $n \equiv b_j \mod{m_j}$ for all $j \in \{ 1, \ldots, s \}$.

In order to formally verify that the output vector 
$b$ is correct we would need to know $n$ as an (long) integer,
that is, the recombined $n$ as a single number.

Since we do not $n$ in this way, we can only provide a criterion
to catch some incorrect results. Here's one such criterion.
This criterion will accept all correct results, but may
fail to detect some incrorrect ones. However, this is very unlikely.


The vector $(b_1 \ldots, b_s)$ satisfies the relation
\begin{equation*}
n = b_1 + b_2 M_2 + \cdots b_s M_s
\end{equation*}
where $M_j$ is the product of $m_1,\ldots,m_{j-1}$.

Therefore we have
$n \equiv b_1 \mod{m_1}$ and thus we have
$x_1 \equiv b_1 \mod{m_1}$.
Next we have
$n \equiv b_1 + b_2 M_2 \mod{m_2}$ and thus we have
$x_2 \equiv b_1 + b_2 M_2  \mod{m_2}$.
More generally, for all $j \in \{ 1, \ldots, s \}$ we have
$x_j  \equiv b_1 + b_2 M_2 + \cdots + b_j M_j$.

The following pseudo-C function implements this criterion

int is\_probably\_correct (s,b,x,m)
{
    for (j=1,j<=s,j++) { \\
        M = 1;\\
        S = b\_1;\\
        for (k=2,k<=j;k++) {\\
           M = (M * m\_{k-1}) \% (m\_j);\\
           S  = (S + (b\_k * M) \% m\_j) \% m\_j;
        }\\
        if (x\_j - S) \% m\_j != 0 {  return 1; }\\
     }
     return 0;
}

\end{document}