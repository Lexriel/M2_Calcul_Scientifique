\subsection{Divide and Conquer Cuda code}
\subsubsection{taylor\_shift.cu}
%\lstinputlisting{code/taylor_shift.cu}

\subsubsection{taylor\_shift\_cpu.cu}


\subsubsection{taylor\_shift\_kernel.cu}


\subsubsection{taylor\_shift\_fft.cu}


\subsubsection{taylor\_shift\_conf.h}

\subsubsection{File calling these procedures : testGPU.cu}

\subsection{Horner's method C++ code}

\subsection{Divide and Conquer C++ code}

\subsection{Maple code}
The procedure we describe in this part makes the taylor shift by one of a polynomial $pp$ of variable $x$ in input modulo $prime$ which is a prime number. We can create a random polynomial $pp$ with the maple instruction \textit{randpoly}, for example :\\
\textit{pp := randpoly($x$, degree = $2^{10} - 1$, terms = $2^{10}$):}\\

Then we can run the following procedure with $pp$ and a prime number we choose ($958922753$ as for the other examples). Here is the procedure : \\

\lstinputlisting{code/xplus1.input}

