
\begin{frame}
%%\frametitle{Background: fast polynomial arithmetic over finite fields}
\frametitle{Background}
\begin{block}{Goal}
Accelerate dense polynomial arithmetic using GPUs   
\end{block}
\begin{block}{Polynomial multiplication over finite fields}
%%The building block of symbolic computation
\begin{itemize}
\item many algorithms rely on polynomial multiplications
\item \textcolor{red}{modular methods} reduce the computations to finite fields
\item fast algorithms like FFT exist
\end{itemize}
\end{block}
\begin{block}{Fast Fourier Transform (FFT) over finite fields}
$$f \times g = \FFT^{-1} (\FFT(f) \cdot \FFT(g))$$
Challenges comparing to $\FFT$s over \textcolor{red}{complex numbers}:
\begin{itemize}
\item radix 2 $\FFT$ is desirable, keeping other primes invertible 
\item lack of primitive roots of unity, (when the degree is high)
\item modular multiplication is expensive
\end{itemize}
\end{block}
\end{frame}

%% ============================================================================
%% ============================================================================
\begin{frame}
\frametitle{Discrete Fourier Transform (DFT)}



\begin{block}{Definition}
Given a primitive $n$-th root of unity $\omega$ (i.e. $\omega^{n/2} = -1$), 
and $$f(t) = x_0 + x_1t + \cdots + x_{n-1}t^{n-1},$$
${\mathrm{{DFT}}}_n^{\omega}(f) $ is $\mathbf{y} = (y_0, \ldots, y_{n-1})$ with 
\textcolor{blue}{$y_k = f(\omega^{k})$ for $0 \leq k < n$}. 
As a matrix-vector product, it is
\begin{equation}
\mathbf{y} = {\mathrm{DFT}}_n\, \mathbf{x},\quad 
\mathrm{DFT}_n = [\omega^{k\ell}]_{0\leq k,\, \ell < n}. 
\end{equation}
\end{block}


\begin{block}{Example}
$$\left\{ 
    \begin{array}{ccc} 
        y_0 & = & x_0 + x_1 \\ 
        y_1 & = & x_0 - x_1 
  \end{array}\right.
\Longleftrightarrow
\begin{bmatrix} y_0 \\ y_1 \end{bmatrix} 
= \begin{bmatrix} 1 & 1 \\ 1 & -1
\end{bmatrix}
\begin{bmatrix} x_0 \\ x_1 \end{bmatrix}$$
That is, $\DFT_2 = \left[\begin{array}{cr} 1 & 1\\ 1 & -1 \end{array}\right]$.
\end{block}

\end{frame}
%% ============================================================================

%% ============================================================================
\begin{frame}
\frametitle{FFT-based multiplication}

\textcolor{darkred}{M(d)}  number of coefficient operations in degree less than  \textcolor{darkred}{d}.

\begin{tabular}[t]{|l|l|}
\hline
Classical Multiplication & \textcolor{darkred}{$\M(d)~=~2d^2$}\\
\hline
Karatsuba Multiplication & \textcolor{darkred}{$\M(d)~=~9d^{1.59}$} \\
\hline
FFT over appropriate ring & \textcolor{darkred}{$\M(d)~=~9/2 d\,\log d + 3 \, d$} \\  \hline
\end{tabular}

\bigskip 

\fbox{
\begin{minipage}{10 cm}
\begin{description}
\item[{\bf Input:}] $f, g \in {\K}[x]$ and ${\omega}$ a $s$-primitive
                    root of unity for $s > {\deg}(f) + {\deg}(g)$
                    and $s$ is a power of $2$.
\item[{\bf Output:}] the product $fg$
\end{description}
\begin{tabbing} 
\quad \= \quad \= \quad \= \quad \kill
(1) \> \> Evaluate $f$ and $g$ at ${\omega}^i$ for $i = 0 \cdots s-1$ \\
(2) \> \> Evaluate $fg$ at ${\omega}^i$ for $i = 0 \cdots s-1$ \\
(3) \> \> Interpolate and {\bf return} $fg$
\end{tabbing}
\end{minipage}
}
See (M.M.M. Yuzhen Xie 2009) for implementation techniques.

\end{frame}
%% ============================================================================

%% ============================================================================
\begin{frame}
\frametitle{The fast division trick (1/2)}


\begin{itemize}
\item Let $a, b \in {\A}[x]$ with $n := {\deg}(a) \geq m := {\deg}(b) > 0$,
\red{$b$ monic} and {\A} any commutative ring with 1.
\medskip
\item<2-> We want the \blue{\sf  quotient} $q$ 
and the \blue{\sf  remainder} $r$ of $a$ w.r.t. $b$:
\begin{center}
$a(x) \ = \ q(x) \, b(x) + r(x)$
\end{center}
\medskip
\item<3-> Replacing $x$ by $1/x$ and multiplying the equation by $x^n$:
$$x^n \, a(1/x) \ = \ \left(x^{n-m} q(1/x) \right) \ 
                    \left(x^m \, b(1/x) \right) \ + \ x^{n-m+1} \left( x^{m-1} \,  r(1/x) \right) $$
That is:
\begin{center}
\red{$ {\rm rev}_n (a) \ = \ {\rm rev}_{n-m}(q) \ {\rm rev}_m (b) +  x^{n-m+1} \ {\rm rev}_{m-1}(r) $}
\end{center}
\medskip
\item<4-> Computing 
\red{$( {\rm rev}_m (b))^{-1} \mod{  x^{n-m+1}}$} 
is a \blue{\sf truncated inverse of a power series}.
(S. Cook, 1966) (H. T.  Kung, 1974) and (M. Sieveking, 1972)
\end{itemize}

\end{frame}
%% ============================================================================
\ifshow
\fi

%% ============================================================================
\begin{frame}
\frametitle{The fast division trick (2/2)}

\fbox{
\begin{minipage}{10 cm}
\begin{description}
\item[{\bf Input:}] $f \in {\A}[x]$ such that $f(0) = 1$ and ${\ell} \in {\N}$.
\item[{\bf Output:}] $g \in {\A}[x]$ such that $f \, g \equiv 1 \mod{ \ x^{\ell}}$
\end{description}
\begin{tabbing} 
\quad \= \quad \= \quad \= \quad \kill
\> $g_0$ := $1$ \\
\> $r$ := $\lceil {\log}_2({\ell}) \rceil$ \\
\> {\bf for} $i = 1 \cdots r$ {\bf repeat} \\
\> \> $g_i$ := $\left( 2 g_{i-1} - f \, {g_{i-1}}^2 \right) \ \mod{ \ x^{2^i}} $ \\
\> {\bf return} $g_r$
\end{tabbing}
\end{minipage}
}

\medskip

\begin{itemize}
\item  This algorithm runs in $3 {\M}(\ell) + 0(\ell)$ 
operations in ${\A}$. 
\item Improved versions run in $2\,{\M}(\ell) + O({\ell})$ 
operations in ${\A}$. 
\item Finally, the \blue{\sf  quotient} $q$ 
and the \blue{\sf remainder} $r$ are computed 
in $3 \, {\M}(n - m) + {\M}({\max}(n-m,m)) + {\cal O}(n)$
operations in ${\A}$
\item {\em Modern Computed Algebra} (Gathen Gerhard 99)
\end{itemize}

\end{frame}
