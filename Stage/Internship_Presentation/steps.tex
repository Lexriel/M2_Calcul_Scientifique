%%%%%%%%%%%%%%%%%%%%%%%%%%%%%%%%%%%%%%%%%%%%%%%%%%%%%%%%%%%%%%%%%
\begin{frame}[fragile]
\frametitle{\textbf{\textcolor{orange}{Divide \& Conquer method}}}

Recall how we can realize the \textit{D{\&}C method}.

\begin{block}{}
\textcolor{blue}{\textbf{The Divide \& Conquer method consists of :}}

\begin{enumerate}
\item \textcolor{violet}{split} the polynomial as : $P(x) = P^{(0)}(x+1)+(x+1)^{n/2}\times P^{(1)}(x+1)$
\item \textcolor{violet}{evaluate} $P^{(0)}(x+1)$ and $P^{(1)}(x+1)$ \textcolor{violet}{recursively}
\item \textcolor{violet}{compute a product} when $P^{(1)}(x+1)$ is evaluated
\item \textcolor{violet}{compute a sum} when $P^{(0)}(x+1)$ and the product are evaluated
\end{enumerate}
\end{block}

These are the four main things to implement to realize the \textit{Taylor shift by $1$}.\\
We will only focus on the multiplication, which is the most hard and tricky operation to realize our code.

\end{frame}

%%%%%%%%%%%%%%%%%%%%%%%%%%%%%%%%%%%%%%%%%%%%%%%%%%%%%%%%%%%%%%%%%

%%%%%%%%%%%%%%%%%%%%%%%%%%%%%%%%%%%%%%%%%%%%%%%%%%%%%%%%%%%%%%%%%
\begin{frame}[fragile]
\frametitle{\textbf{\textcolor{orange}{Multiplication : concept}}}

\begin{block}{}
\begin{itemize}
\item \textcolor{blue}{\textbf{Small sizes ($n \leq 512$) :}}\\
\hspace{4mm}
We can use a procedure called \textit{list\_Plain\_Mul} which computes a list of pairwise products of polynomials of \underline{same size}.
\item \textcolor{blue}{\textbf{Big sizes ($n > 512$) :}}\\
\hspace{4mm}
We can use \textit{FFT} whic several procedures which compute a list pairwise products of polynomials of \underline{same size}.
\end{itemize}
\end{block}

\begin{block}{}
\textcolor{violet}{\textbf{Polynomials considered at step $i$ :}}\\
Polynomials \textcolor{red}{$P^{(1)}$} (of size $2^{i}$) and \textcolor{red}{$(x+1)^{2^i}$} (of size $2^{i} + 1$).
\end{block}

\end{frame}
%%%%%%%%%%%%%%%%%%%%%%%%%%%%%%%%%%%%%%%%%%%%%%%%%%%%%%%%%%%%%%%%%

%%%%%%%%%%%%%%%%%%%%%%%%%%%%%%%%%%%%%%%%%%%%%%%%%%%%%%%%%%%%%%%%%
\begin{frame}[fragile]
\frametitle{\textbf{\textcolor{orange}{Multiplication : a challenge}}}

\begin{block}{}
\textcolor{blue}{\textbf{Problems :}}
\hspace{3mm}
\begin{itemize}
\item Polynomials considered \underline{must be of the same size} (rather $2^{i}$) so as to use the procedures of the lab.
\item Size of the product of two such polynomials is not a power of $2$ (but $2^{i+1} - 1$).
\end{itemize}
\end{block}

\begin{block}{}
\textcolor{red}{\textbf{Solutions :}}
\hspace{3mm}
\begin{itemize}
\item Modify procedure \textit{list\_Plain\_Mul} and adapt it for my case.
\item Consider another product with polynomials of same sizes.
\end{itemize}
\end{block}

\end{frame}
%%%%%%%%%%%%%%%%%%%%%%%%%%%%%%%%%%%%%%%%%%%%%%%%%%%%%%%%%%%%%%%%%

%%%%%%%%%%%%%%%%%%%%%%%%%%%%%%%%%%%%%%%%%%%%%%%%%%%%%%%%%%%%%%%%%
\begin{frame}[fragile]
\frametitle{\textbf{\textcolor{orange}{Multiplication : decomposition}}}

We can decompose the product desired as follows :

\begin{block}{}
\footnotesize
\begin{align*}
P^{(1)}(X) \times (X+1)^{2^i} 
&= \left( \sum_{i=0}^{2^i-1} a_i\,X^i\right) \times (X+1)^{2^i}\\
&= P^{(1)}(X) \times \left[(X+1)^{2^i} - 1 + 1 \right]\\
&= P^{(1)}(X) \times \left[(X+1)^{2^i} - 1\right] + P^{(1)}(X)\\
&= P^{(1)}(X) \times X \times \cfrac{(X+1)^{2^i} - 1}{X} + P^{(1)}(X)\\
&= X \cdot \left( P^{(1)}(X) \times \cfrac{(X+1)^{2^i} - 1}{X} \right) + P^{(1)}(X)
\end{align*}
\end{block}

\end{frame}
%%%%%%%%%%%%%%%%%%%%%%%%%%%%%%%%%%%%%%%%%%%%%%%%%%%%%%%%%%%%%%%%%

%%%%%%%%%%%%%%%%%%%%%%%%%%%%%%%%%%%%%%%%%%%%%%%%%%%%%%%%%%%%%%%%%
\begin{frame}[fragile]
\frametitle{\textbf{\textcolor{orange}{Multiplication : technic}}}
\begin{tsmc}
Using the following formula :
$$\textcolor{red}{P^{(1)}(X) \times (X+1)^{2^i}  = X \cdot \left( P^{(1)}(X) \times \cfrac{(X+1)^{2^i} - 1}{X} \right) + P^{(1)}(X)}$$

The multiplication desired amounts to :

\begin{enumerate}
\item \textcolor{violet}{multiplying} $P^{(1)}(X)$ by $[(X+1)^{2^i} - 1]/X$ of sizes $2^i$,
\item doing a \textcolor{violet}{right shift} (multiplication by $X$), and
\item \textcolor{violet}{semi-adding} the result of the two first steps with $P^{(1)}(X)$.
\end{enumerate}
\end{tsmc}

We will just detail how to do the multiplication for the two cases of polynomial sizes.

\end{frame}
%%%%%%%%%%%%%%%%%%%%%%%%%%%%%%%%%%%%%%%%%%%%%%%%%%%%%%%%%%%%%%%%%

%%%%%%%%%%%%%%%%%%%%%%%%%%%%%%%%%%%%%%%%%%%%%%%%%%%%%%%%%%%%%%%%%
\begin{frame}[fragile]
\frametitle{\textbf{\textcolor{orange}{Multiplication : arrays considered (artificial ex.)}}}

If we consider polynomials at step $1$ :

\tiny
\begin{center}
\textit{Polynomial\_shift\_device}[0] = \begin{tabular}{|c|c||c|c||c|c||c|c|}
\hline
1 & 2 & 3 & 4 & 5 & 6 & 7 & 8 \\
\hline
\end{tabular}
\end{center}

\normalsize
Then for the two multiplication techniques used, we consider :

\begin{block}{}
\begin{itemize}
\item \textcolor{blue}{\textbf{"Small sizes" ($n \leq 512$) :}}\\
\hspace{4mm}
\tiny
\begin{center}
\textit{Mgpu} = \begin{tabular}{|c|c||c|c||c|c||c|c|}
\hline
3 & 4 & 2 & 1 & 5 & 6 & 2 & 1 \\
\hline
\end{tabular}
\end{center}
\hspace{4mm}

\item \normalsize{\textcolor{blue}{\textbf{"Big sizes" ($n > 512$) :}}}\\
\hspace{4mm}
\tiny
\begin{center}
\textit{fft\_device} = \begin{tabular}{|c|c|c|c||c|c|c|c||c|c|c|c||c|c|c|c|}
\hline
3 & 4 & 0 & 0 & 2 & 1 & 0 & 0 & 5 & 6 & 0 & 0 & 2 & 1 & 0 & 0 \\
\hline
\end{tabular}
\end{center}
\end{itemize}
\end{block}

\end{frame}
%%%%%%%%%%%%%%%%%%%%%%%%%%%%%%%%%%%%%%%%%%%%%%%%%%%%%%%%%%%%%%%%%

%%%%%%%%%%%%%%%%%%%%%%%%%%%%%%%%%%%%%%%%%%%%%%%%%%%%%%%%%%%%%%%%%
\begin{frame}[fragile]
\frametitle{\textbf{\textcolor{orange}{Multiplication according to the size of the polynomials}}}

\begin{block}{}
\begin{itemize}
\item \textcolor{blue}{\textbf{"Small sizes" ($n \leq 512$) :}}\\
\hspace{4mm}
We use \textit{Mgpu} and multiply directly pairwise polynomials inside, then do a right shift. Product size is still a power of $2$. So \textit{list\_Plain\_Mul\_and\_right\_shift} is used to do : $X \cdot \left( P^{(1)}(X) \times [(X+1)^{2^i} - 1]/X \right)$.
\end{itemize}
\end{block}

\begin{block}{}
\begin{itemize}
\item \normalsize{\textcolor{blue}{\textbf{"Big sizes" ($n > 512$) :}}}\\
\hspace{4mm}
We transform \textit{Mgpu} in \textit{fft\_device} and then use \textit{FFT} for the multiplication $P^{(1)}(X) \times [(X+1)^{2^i} - 1]/X$.
\end{itemize}
\end{block}

The \textit{FFT} will be explained in the following section.

\end{frame}
%%%%%%%%%%%%%%%%%%%%%%%%%%%%%%%%%%%%%%%%%%%%%%%%%%%%%%%%%%%%%%%%%
