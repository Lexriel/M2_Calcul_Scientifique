%%%%%%%%%%%%%%%%%%%%%%%%%%%%%%%%%%%%%%%%%%%%%%%%%%%%%%%%%%%%%%%%%
\begin{frame}[fragile]
\frametitle{\textbf{\textcolor{orange}{Possible improvements \& FFT primes}}}

\begin{block}{}
\begin{itemize}
\item We need to improve the parallel computation of the sequence $(i!)_{1 \leq i \leq n}$, and
\item We can reduce the size of the array storing the elements of $(x+1)^{2^i}$ as these polynomials are symmetric.
\end{itemize}
\end{block}

\begin{block}{}
The \textit{Taylor shift} modulo $p$ is needed several times to get the \textit{Taylor shift} in $\mathbb{Z}$. Two questions arises :
\vspace{3mm}
\begin{enumerate}
\item \textcolor{violet}{What prime numbers must we use ?}
\item \textcolor{violet}{How many such primes numbers must we use ?}
\end{enumerate}
\end{block}

\begin{block}{}
Primes numbers of the form $\textcolor{red}{p = M \times 2^j + 1}$ yield the best performance for \textit{FFT}, with $p>n$ and $M<2^j$ odd integer.
\end{block}


\end{frame}
%%%%%%%%%%%%%%%%%%%%%%%%%%%%%%%%%%%%%%%%%%%%%%%%%%%%%%%%%%%%%%%%%

%%%%%%%%%%%%%%%%%%%%%%%%%%%%%%%%%%%%%%%%%%%%%%%%%%%%%%%%%%%%%%%%%
\begin{frame}[fragile]
\frametitle{\textbf{\textcolor{orange}{Combination with the Chinese Remainder thm.}}}

\begin{block}{}
We need to Taylor shift by a prime number $(m_i)_{1 \leq i \leq s}$ $s$ times.\\
For each coefficient in $\mathbb{Z}$, we obtain a vector $\textbf{x} = (x_1,\dots,x_s)$.
\end{block}

\begin{block}{}
\textcolor{blue}{\textbf{Objective :}} using the \textit{CRT2}, compute the image $a$ of $\mathbf{x}$ by
\begin{center}
$\mathbb{Z}/m_1\mathbb{Z} \times \cdots \times \mathbb{Z}/m_s\mathbb{Z} \cong \mathbb{Z}/m_1\cdots m_s\mathbb{Z}$.\\
\end{center}
\vspace{3mm}
\textcolor{blue}{\textbf{Representation :}} we can represent $a$ by $\textbf{b} = (b_1,\dots,b_s)$ s.t.
\begin{center}
$\textcolor{red}{a = b_1 + b_2m_2 + b_3m_1m_2 + \cdots + b_sm_1\cdots m_{s-1}}$.
\end{center}
\end{block}

Then we will use the conversion of modular numbers to their \textcolor{violet}{\textbf{mixed radix representation by a matrix formula}} to compute $\textbf{b}$.

\end{frame}
%%%%%%%%%%%%%%%%%%%%%%%%%%%%%%%%%%%%%%%%%%%%%%%%%%%%%%%%%%%%%%%%%

%%%%%%%%%%%%%%%%%%%%%%%%%%%%%%%%%%%%%%%%%%%%%%%%%%%%%%%%%%%%%%%%%
\begin{frame}[fragile]
\frametitle{\textbf{\textcolor{orange}{Definition (1)}}}

Let us consider $m_1, m_2,\dots,m_s$ distinct prime numbers.

\begin{Definition}[$(m_{i,j})$ and $(n_{i,j})$]
We define the sequences $\textcolor{red}{(m_{i,j})_{1 \leq i < j \leq s}}$ and $\textcolor{red}{(n_{i,j})_{1 \leq i < j \leq s}}$ such that :
$$\left\{
\begin{array}{l}
m_{i,j}\times m_i \equiv 1\mod m_j \;\; | \;\; 0 \leq m_{i,j} < m_j\\
n_{i,j} = m_j - m_{i,j}
\end{array}
\right.$$
\end{Definition}

\end{frame}

%%%%%%%%%%%%%%%%%%%%%%%%%%%%%%%%%%%%%%%%%%%%%%%%%%%%%%%%%%%%%%%%%
\begin{frame}[fragile]
\frametitle{\textbf{\textcolor{orange}{Definition (2)}}}

\begin{Definition}[matrix $A$]
We define the matrix $(A_k)_{1 \leq k \leq s-1}$ as the following :
$$\textcolor{darkgreen}{A_k = \left(
  \begin{array}{c|c}
     I_{k-1}  & 0    \\ \hline
     0        & B_k
  \end{array} \right)} \mbox{ with } $$

$$\textcolor{darkgreen}{B_k = \left(
  \begin{array}{c c c c c}
     1      & n_{k,k+1} & n_{k,k+2} & \dots  & n_{k,s} \\
     0      & m_{k,k+1} & 0         & \dots  & 0       \\
     \vdots & \ddots    & m_{k,k+2} & \ddots & \vdots  \\
     \vdots &           & \ddots    & \ddots & 0       \\
     0      & \dots     & \dots     & 0      & m_{k,s}
  \end{array} \right)}
$$
\end{Definition}

\end{frame}
%%%%%%%%%%%%%%%%%%%%%%%%%%%%%%%%%%%%%%%%%%%%%%%%%%%%%%%%%%%%%%%%%

%%%%%%%%%%%%%%%%%%%%%%%%%%%%%%%%%%%%%%%%%%%%%%%%%%%%%%%%%%%%%%%%%
\begin{frame}[fragile]
\frametitle{\textbf{\textcolor{orange}{Mixed radix representation by a matrix formula}}}

\begin{Theorem}
$$\textcolor{red}{\textbf{b} = \left(\dots\left(\left(\left( \textbf{x}A_1 \right)A_2\right)A_3\right)\dots\right)\ A_{s-1}}$$
\end{Theorem}

\begin{Definition}
As $A_k$ is sparse, we don't really need to multiply our results by a matrix.\\
Thus, we will consider sequences $(L_k)_{1 \leq k \leq s-1}$ and $(D_k)_{1 \leq k \leq s-1}$ respectively the first row of $A_k$ and the diagonal of $A_k$ such that :
$$\textcolor{darkgreen}{(L_k) = (n_{k,j} \; | \; k+1 \leq j \leq s)},$$
$$\textcolor{darkgreen}{(D_k) = (m_{k,j} \; | \; k+1 \leq j \leq s)}.$$
\end{Definition} 

This gives the following algorithm (with $d=n-1 = 2^e-1$) :

\end{frame}
%%%%%%%%%%%%%%%%%%%%%%%%%%%%%%%%%%%%%%%%%%%%%%%%%%%%%%%%%%%%%%%%%

%%%%%%%%%%%%%%%%%%%%%%%%%%%%%%%%%%%%%%%%%%%%%%%%%%%%%%%%%%%%%%%%%
\begin{frame}[fragile]
\frametitle{\textbf{\textcolor{orange}{Algorithm for the mixed radix representation}}}

\begin{block}{}
\underline{\textbf{Input :}} $\textbf{X}[0..d][1..s]$, $s$, $(m_i)_{1\leq i \leq s}$, $(L_k)_{1 \leq k \leq s-1}$, $(D_k)_{1 \leq k \leq s-1}$ \\
$\textbf{Y} := \textbf{X}$\\
for $k = 1..s-1$ do \\
\hspace{4mm}for $i = 0..d$ do \\
\hspace{8mm}for $j = k+1..s$ do \\
\hspace{12mm}$Y_{i,j} := \left[ \left( Y_{i,j}L_{k,j}\mod m_j \right) + \left( Y_{i,j}D_{k,j}\mod m_j \right) \right]\mod m_j$ \\
\hspace{8mm}end do \\
\hspace{4mm}end do \\
end do \\
\underline{\textbf{Output :}} $\textbf{Y}[0..d][1..s]$
\end{block}


This algorithm can be parallelized but the different loops in $k$ need less and less computations, parallelization must me done with reflection.

\end{frame}
%%%%%%%%%%%%%%%%%%%%%%%%%%%%%%%%%%%%%%%%%%%%%%%%%%%%%%%%%%%%%%%%%
